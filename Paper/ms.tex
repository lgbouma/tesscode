\documentclass[preprint]{aastex}

\bibliographystyle{apj}

\def \teff {T_{\rm eff}}

\begin{document}

\title{Simulating the Planet Yield of the Transiting Exoplanet Survey Satellite}

\author{Peter W. Sullivan, Josua N. Winn, Roland K. Vanderspek, George R. Ricker}
%\altaffiltext{1}{MIT-Kavli Institute for Astrophysics and Space Research}

%\begin{abstract}

%\end{abstract}

\section{Introduction}
Discuss Kepler results, shortcomings of ground-based surveys.

\section{Description of TESS}
\subsection{Mission and survey}
Image showing number of pointings vs. ecliptic latitude
\subsection{Optics and CCD}
FOV, EPD, PSF, bandpass, noise

\section{Simulations}
\subsection{Star Catalog}
Luminosity function, counts vs. spectral type
\subsection{Background Model}
Zodiacal light, unresolved stars
\subsection{Planet Distribution}
Fressin rates, plot of radius vs. period
\subsection{Observing model}
Noise model, observed number of transits, detection thresholds

\section{Results}
\subsection{Planet Yields}
\subsection{Habitability of the TESS-detected planets}
Plot of $S$ vs. $\teff$ for detected planets
\subsection{Observations of the Kepler Objects of Interest}
Histogram of transit timing precision

\section{Follow-up of the TESS detections}
\subsection{Photometry}
Plot of transits in ppm/hr versus ($I_C$ and/or $J$) magnitude.
\subsection{Radial Velocity}
Plot of RV $K$ vs. ($I_C$ and/or $V$) magnitude.

%\begin{figure}
%\epsscale{1.0}
%\plotone{filters.eps} 
%\caption{\emph{Left:} Intermediate- and narrow-band filters used to detect LAEs. \emph{Right:} Filters used to detect the Ly-$\alpha$ absorbers. For reference, the composite LBG spectrum from \citet{shap03} is also plotted near the expected redshifts of the LAEs and background galaxies.}
%\label{filter}
%\end{figure}

%\section{Conclusions}

\acknowledgements

%\clearpage
%\bibliography{ms}

\end{document}
