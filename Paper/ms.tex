%\documentclass[12pt,preprint]{aastex}
%\documentclass[preprint2,12pt]{aastex}
\documentclass{emulateapj}
\usepackage{apjfonts}
\usepackage{graphicx}

\shortauthors{Author1, Author2, Author3}
\shorttitle{Finding planets with TESS}

\begin{document}

% ------------------------------------------------------------------------
% New commands
%
\def\ltsima{$\; \buildrel < \over \sim \;$}
\def\lsim{\lower.5ex\hbox{\ltsima}}
\def\gtsima{$\; \buildrel > \over \sim \;$}
\def\gsim{\lower.5ex\hbox{\gtsima}}
\def\tess{{\it TESS} }
\def \teff {T_{\rm eff}}
\def \phir {\Phi_{\rm R}}
\def \fov {24$^{\circ}$}
\def \pixsz {21.1''}
\def \aeff {69.1 cm$^2$ }    
\def \epd {105 mm}                          
                                                                                          
% -------------------------------------------------------------------------
%

\bibliographystyle{apj}

\title{ Searching the Local Neighborhood for Planets with the
  Transiting Exoplanet Survey Satellite }

\author{
Author list TBD
}

% \journalinfo{Draft version}
\slugcomment{To be submitted to {\it The Astrophysical~Journal}}

 \altaffiltext{1}{Department of Physics, and Kavli Institute for
   Astrophysics and Space Research, Massachusetts Institute of
   Technology, Cambridge, MA 02139}

 \altaffiltext{2}{Harvard-Smithsonian Center for Astrophysics, 60
   Garden St., Cambridge, MA 02138}

\begin{abstract}

  We establish the expected properties of nearby transiting planets,
  using a simple model for the properties of nearby stars as well as
  the transit occurrence rates measured by {\it Kepler}. We describe
  the {\it Transiting Exoplanet Survey Satellite} and simulate its
  ability to detect the nearby population of transiting planets. We
  discuss the prospects for follow-up of these planets to measure
  their sizes, masses, and atmospheric properties.

\end{abstract}

\keywords{planets and satellites:\ detection --- space vehicles:\
  instruments --- surveys}

\section{INTRODUCTION}

Ah, transiting planets! We learn so much by studying them.

{\it Kepler} provided a sample of several thousand transiting planets,
out of a fairly well-controlled sample of some 150,000 stars (mainly
FGK but including a few thousand M stars).  The stated objective of
the mission was to determine planet occurrence rates, partly to
understand planet formation and the diversity of planetary systems,
and partly to inform future missions.  The next big objective for the
future missions is to seek the brightest, nearest, or otherwise most
favorable examples of the various types of Kepler planets,
facilitating the study of their masses, atmospheres, orbits, etc.

\tess is an approved NASA mission that aims to achieve this goal.  It
will gather the ``low hanging fruit'' (short period planets) from over
the whole sky. \tess is the first mission that will benefit from the
knowledge that {\it Kepler} has provided.  It will find great targets
for everyone's telescopes, including the {\it JWST}.

This paper is organized as follows:

First, in Section~\ref{sec:stars_and_planets}, we build a model for
the properties of nearby stars and planets, using local stellar
censuses and {\it Kepler} planet occurrence rates. This might be split
off as a separate paper.

Then, in Section~\ref{sec:tess}, we describe the \tess mission.  We
give a brief history and then specify the key characteristics that
affect the ability to detect planets. Since \tess is not yet built, so
all of this information is provisional to some degree.

Section~\ref{sec:yield} presents simulations of the quantity and
characteristics of the transiting planets to be discovered by \tess.
We tally the number of genuine planet detections.  These will be
accompanied by eclipsing binaries and other ``false positives'', but
we will probably leave the problem of calculating the false positive
rate for another day.

Finally, Section~\ref{sec:followup} discusses the prospects for
following up on these detections with other instruments. We will
consider the prospects for mass measurement through Doppler
monitoring, and atmospheric characterization.

\section{NEARBY TRANSITING PLANETS}
\label{sec:stars_and_planets}
There have been many studies that have used Kepler data to determine
planet occurrence rates. This is interesting for helping to understand
planet formation, and for estimating one of the factors in the Drake
equation. However, comparatively little attention has been paid to the
purpose of informing future missions.  What are the expected
properties of the top-50 most favorable transiting super-Earth systems
in the sky? Where should we expect to find the 50 nearest systems? How
bright are the 50 brightest host stars (in the visual band, or an
infrared band)?

None of the envisioned future transit surveys will be as sensitive
than Kepler to small planets, or long-period planets (not the ground
based surveys, not TESS, not PLATO).  This has the effect of
simplifying the problem, because we need not extrapolate beyond the
Kepler detections to smaller or longer-period planets. Kepler
essentially provides perfect knowledge of the type of planets that
will be detected in near-term future transit missions.

The only required information Kepler does not tell us particularly
well (and this is a serious omission) is the distribution of planets
around M dwarfs, which are very numerous in the local neighborhood but
are not very well represented in the Kepler target
catalog. Nevertheless we do our best and live with the larger
uncertainties.

Our approach will be to create a realistic model for the properties of
the ``best'' few hundred thousand stars in the sky. We will state
later what we mean by ``best''. Then we will populate those stars with
planets, using the occurrence rates determined from {\it Kepler}.  We
will then inquire into the distribution of apparent magnitudes and
distances of various interesting types of planetary systems.

It is possible to anticipate the results with the following simple
calculation. The Kepler field is 100 square degrees which is about
1/412th of the entire sky. For the same volume, an all-sky survey
would be shallower in the radial direction by a factor $\sqrt{412} =
20$.  Being closer by a factor of 20 corresponds to being brighter by
3.3~mag.  So to first approximation, everything simply shifts to
brighter magnitudes by about 3~mag.  But there are complications due
to extinction, galactic structure, the peculiar selection criteria
described in Batalha et al.\ (2010), etc.\footnote{Regarding the
  latter, Kepler did not look at all the stars satisfying any
  particular apparent magnitude cut, due to the telemetry constraint.
  About 260,000 stars in the KIC field were identified as good planet
  searching targets. This was reduced to 170,000 by triage according
  to minimum detectable planet size. There are a few thousand OBA
  stars, and about 3000 M stars.} This calls for a Monte Carlo
approach, in which these effects can be taken into account, as well as
making it straightforward to account for uncertainties in the
occurrence rates, false positive probabilities, etc.

\subsection{Stellar population}

Our approach will be to create realistic model for the properties of
the "best" few hundred thousand stars in the sky, where "best" might
mean:

\begin{enumerate}

\item Brightest apparent magnitude in the $V$ band.  Important for
  current Doppler spectrometers, particularly of the iodine
  persuasion.

\item Brightest apparent magnitude in the $I$ band.  Important for
  current Doppler spectrometers, and also (as we will see) important
  for TESS.

\item Brightest apparent magnitude in the $K$ band. Important for JWST
  and occultation spectroscopy.

\item Brightest apparent bolometric magnitude. (Why?)

\item Highest relative SNR of transit signal for a planet of a given
  size. That is, stars are ranked according to $R^{-3/2}
  (L/d^2)^{1/2}$ where $R$ is the stellar radius, $L$ is the
  (bolometric?) luminosity, and $d$ is the distance from Earth.

\item Nearest to the Earth. Important for astrometric detection, direct
  imaging, and spaceship journeys.

\end{enumerate}

We will create a model for the properties and space densities of
nearby stars that allows for the construction of all these types of
samples. For our purpose, the most important thing to get right is the
{\it radius function}. We need to know the number of stars $dN$ in
each bin of apparent magnitude $dm$ (or distance $ds$) that are
between $R$ and $R+dR$ in radius.

We need a simulated sample of at least 200,000 stars to take full
advantage of the size of the Kepler target catalog.

Memo 3 describes an attempt to use local $J$-band censuses, in
conjunction with the Dartmouth models, to derive the radius function.
The model only goes up to one solar radius and is only valid for
unevolved stars. This needs to be improved by
\begin{enumerate}
\item Developing a model that spans all relevant stellar types, at
  least F5-M5.
\item Performing checks against actual catalogs such as RECONS,
  various M dwarf catalogs, and the TESS draft star catalog. Maybe
  also against TRILEGAL.
\end{enumerate}

How should this model be created? We could use a combination of
Hipparcos (for FGK) and local censuses (for M).

For each Hipparcos non-giant (i.e., dwarf or subgiant), we establish a
radius, mass, spectral type, etc., with reference to the
literature.\footnote{Ofek~(2008) has already performed a rudimentary
  KIC-like spectral classification of Hipparcos/Tycho
  stars. Ofek~(2008) also provided synthetic SDSS mags for the Tycho
  stars which could be used (after accounting for reddening) to
  compare to the SDSS colors of the Kepler stars.}  We also compute
the distance out to which such a star would have been included in the
Hipparcos magnitude-limited sample. This distance will depend on the
absolute Hipparcos magnitude. Then, in creating a simulated catalog,
we randomly drop that exact type of star within the search volume. Or,
if a larger catalog is desired, we drop 8 stars randomly within a
volume 8 times as large as the Hipparcos volume.

This will fail for stars that are too optically faint for Hipparcos to
have detected very many, i.e., M dwarfs. For the M dwarfs we could
continue to rely on local census information, as long as we are also
satisfied that our model agrees with the actual star catalogs that are
available.

Another possible approach would be to use TRILEGAL for the whole
sample.

\subsection{Planet population}

We populate the simulated stars with simulated planets based on the
Kepler occurrence rates.

We could use the occurrence rates of Fressin et al.\ (2013), Dressing
\& Charbonneau~(2013), or Petigura et al.\ (2013).

Or we could take a more straightforward approach and base the
calculation directly upon Kepler transit detection rates: simply
assign random KIC numbers to the simulated stars, and assign to that
simulated star the same exact planetary system that was seen around
that KIC star (if any).  This approach does not take into account
false positives, so we might simply reject transit detections with a
certain false positive probability provided by Fressin et al.\ (2013).

We plot our model for the occurrence rate distribution in the
period/radius plane, for each type of star (FGKM).  Will we allow the
distribution to differ among FGK stars?  How different is it for M
stars?

\begin{figure}[htbp]
\epsscale{1.2}
\plotone{fress.eps}
\caption{We adopt the planet occurrence rates as a function of radius and period from \citet{fressin2013}, which are calculated from the \textit{Kepler} mission. Within the coarse bins in the radius-period plane (\textit{upper left panel}), we assume the periods are distributed according to $P^{-1}$, giving a flat distributions in $\log{P}$ (\textit{lower left panel}). For planet radii $R_p>1.25R_{\oplus}$, we assume the radii are distributed by $R_p^{-1.7}$, and for $R_p<1.25R_{\oplus}$, we assume a flat distribution in $R_p$. This gives a fairly smooth distribution in $R_p$ (\textit{upper right panel}).}
\end{figure}

\subsection{Best and brightest}
 
We plot each planet in ``detectability space'': the horizontal axis is
apparent magnitude of host star, and the vertical axis is the minimum
noise level needed to detect the planet with a specified SNR based on
observing for one hour (or some other fixed amount of time).  This
tells us the required noise characteristics of any instrument that
will be used to detect the planets.

We also look in period/radius space, and inquire into the apparent
magnitude distribution of the stars in each bin; what is the expected
magnitude of the brightest, 10th brightest, 100th brightest star?  And
the same, for distance rather than apparent magnitude.

We will specifically consider hot Jupiters (we should recover the
known result that $V\approx 8$ for brightest); super-Earths;
Earth-sized planets; multis with more than one coplanar planet; and HZ
planets.  We will work out the results for transiting planets, and for
non-transiting planets.

\section{THE TRANSITING EXOPLANET SURVEY SATELLITE}
\label{sec:tess}
\subsection{History and overview}

The basic idea of a wide field space based transit survey originated
around 2005, a few years after Ground-Based Transit Fever, and several
years before Kepler launched. In the beginning was HETE-2. Then we
tried SMEX. And at last, EX.  We were approved in April 2013. Launch
is scheduled for late 2017.

It will use 4 CCD cameras to tile the sky over 2 years.

\subsection{Cameras and detectors} 

\tess employs four refractive cameras. Each camera has a field-of-view (FOV) of \fov, and the fields are stacked in ecliptic latitude to give the observatory a 24x96$^{\circ}$ FOV. A mosaic of four CCDs lie at the focal plane of each camera with a total of 4096x4096 pixels. The projected pixel size on the sky is \pixsz. In order to minimize the contamination of background flux in the photometric apertures, the point-spread function (PSF) of th lens must concentrate the majority of the flux from a star to within a few pixels. 

The entrance pupil diameter (EPD), which describes the physical collecting area of the lens before losses are taken into account, is \epd.

We choose a bandpass from 0.6 to 1.0 micron. We wanted to operate in
the optical to take advantage of the desirable properties and heritage
of CCDs. But we wanted to move as far as possible toward the red,
where M dwarfs are relatively brighter.

Deep-depletion CCDs manufactured by MIT-Lincoln Laboratry will be used for their high quantum efficiency at near-infrared wavelengths.

Since the lens must deliver a small PSF over a large focal plane and over a wide bandpass, the lens requires seven lenses.


\subsection{Spacecraft and orbit} 
\tess will conduct is observations from a unique orbit in a 2:1 resonance with the moon. The 13.7-day period allows long observing durations with high thermal stability.
The spacecraft is solar-powered, so the orbit will be inclined from the ecliptic plane to eliminate long periods of eclipse. 
A combination of thruster firings and a gravitational assist from the moon will raise the orbit and change the inclination from the initial orbit delivered by the launch vehicle to the final orbit from which \tess will conduct is observations.
Three-axis stabilization will be accomplished with reaction wheels, and fine pointing will be derived from the science cameras.

\begin{figure}[ht]
\epsscale{1.0}
\plotone{npnt.eps}
\caption{The number of 26.7-day pointings TESS spends on a given star is a function of its ecliptic coordinates. Here, the number of pointings are shown in a polar ecliptic projection; the pointings in the northern and southern ecliptic hemispheres are symmetric. Coverage near the ecliptic (dashed line) is sacrificed in favor of coverage near the ecliptic poles, which receive continuous coverage for 360 days.}
\label{fig:npoint}
\end{figure}


\subsection{Survey} 

Schedule of telescope pointings. Image showing sky coverage, and
overlap regions. Number of pointings vs.\ ecliptic latitude.

Cadence of observations. Postage stamps of catalog stars, and full
frame images.  How the data are returned to Earth.

\section{SIMULATIONS OF THE TESS YIELD}
\label{sec:yield}
\subsection{Star Catalog}

We sample from the model developed in Section 1.  Specifically, the
star sample is based on an $I$-band selection but with a sliding
scale, optimized for planet detectability.

\begin{figure*}[ht]
\epsscale{1.0}
\plotone{npix.eps}
\caption{The TESS optics serve two purposes: to focus the light from a target star into few pixels to reject the light from nearby stars. On the left, we plot the fraction of accumulated flux over the number of pixels in the photometric aperture. On the right, we plot the point source rejection ratio (PSRR), expressed as a difference in astronomical magnitudes, over the distance between a target pixel and a contaminating star.}
\end{figure*}

\subsection{Noise model}

\begin{figure*}[ht]
\epsscale{1.0}
\plotone{bgmap.eps}
\caption{We model the zodiacal light as a function of eclipic coordinates (\textit{upper left}) and the background stellar flux as a function of galactic coordinates (\textit{upper right}). The lower panels show the combined surface brightness.}
\end{figure*}

Photon counting noise. Read noise. Background noise from faint stars, and zodiacal
light. Pointing jitter.

Present noise in one hour versus apparent magnitude, an ``exposure
time calculator''. Compare to the ``detectability space'' plot that
was made earlier.


\subsection{Detection model}

Establish the minimum SNR threshold based on the statistical false
positive rate (in the neighborhood of 7). We require 3 transits for
secure detection but we will also keep track of single and double
transit detections; they will just require more work to be confident
we know what we are dealing with.

\subsection{Results}

\begin{figure}[ht]
\epsscale{1.0}
\plotone{hz.eps}
\caption{The inner (red) and outer (blue) radii of the habitable zone are defined by the insolation $S$ (relative to the Earth-Sun value) and may depend weakly on the effective temperature of the star. TESS is far more sensitive to planets with periods shorter than the habitable zone, but it should still find several planets in the habitable zone of cooler stars.}
\end{figure}


Radius/apparent magnitude space, currently known planets and TESS
planets.
 
Radius/period space.

Plot of $S$ vs. $\teff$ for detected planets, which ones are
habitable?

\section{PROSPECTS FOR FOLLOW-UP}
\label{sec:followup}
We discuss prospects for follow-up observations based on the expected
characteristics of the TESS stars and planets.

Photometric follow-up: to get better transit light curves, and measure
wavelength dependent transit depths.  We can plot various telescope
sensitivity curves in the space of noise versus apparent magnitude.

Radial velocity. We can compute relative HARPS time needed to detect
planetary signal. Plot RV amplitude vs $V$ mag

Atmospheric characterization. We compute relative JWST time needed to
detect the atmosphere. If this too complicated, just show the $K$ mag of small-planet hosts below some insolation value.

\begin{enumerate}
\item How many super-Earths are there on the sky with host stars bright
   enough for present-day RV spectroscopy?
\item Same but for space-based atmospheric characterization.
\item Same but for ... other applications.
\end{enumerate}

%\begin{figure*}[ht]
%
% \begin{center}
% \leavevmode
% \hbox{
% \epsfxsize=6.5in
% \epsffile{phot.eps}}
% \end{center}
% \vspace{-0.2in}

%%\epsscale{1.0}
%%\plotone{phot.eps}

% \caption{ {\bf {\it Kepler} photometry.} 
%   {\it Top.}---Time series of relative flux. Eclipse data have been
%   removed. Vertical dotted
%   lines are plotted every 35.1~days, the estimated rotation period.
%   Different colors indicate the 4 segments for which periodograms
%   were computed separately in order to gauge the uncertainty in the rotation period.
%   {\it Bottom.}---Lomb-Scargle periodograms of the entire time
%   series (thick black line) and each of the 4 segments
%   (thin colored lines).
%   \label{fig:phot}}%

%\end{figure*}

% \begin{figure*}[ht]
% 
% \begin{center}
%  \leavevmode
%  \hbox{
%  \epsfxsize=6.5in
%  \epsffile{rv.eps}}
%  \end{center}
%  \vspace{-0.2in}
% 
% %\epsscale{1.0}
% %\plotone{rv.eps}
% 
%  \caption{ {\bf Keck radial velocities.} 
%    {\it Top.}---Apparent radial velocity (solid points) and the
%    best-fitting model (gray curve). {\it Bottom.}---After subtracting
%    the best-fitting orbital model, thereby isolating the RM anomaly.
%    Each night's data is shown separately, along with the residuals.
%    \label{fig:rv}}
% \end{figure*}

\acknowledgments
We thank Juan Carlos Torres, the M.I.T.\ School of Science, and other
generous patrons of TESS.

\eject
\clearpage

\begin{thebibliography}{}

\bibitem[Binney \& Merrifield(1998)]{1998gaas.book.....B} Binney, J.,
  \& Merrifield, M.\ 1998, Galactic astronomy (Princeton University
  Press)

\bibitem[Bochanski et al.(2010)]{2010AJ....139.2679B} Bochanski,
  J.~J., Hawley, S.~L., Covey, K.~R., et al.\ 2010, \aj, 139, 2679

%\bibitem[Boyajian et al.(2012)]{2012ApJ...757..112B} Boyajian, T.~S.,
%  von Braun, K., van Belle, G., et al.\ 2012, \apj, 757, 112

\bibitem[Covey et al.(2008)]{2008AJ....136.1778C} Covey, K.~R.,
  Hawley, S.~L., Bochanski, J.~J., et al.\ 2008, \aj, 136, 1778

\bibitem[Cruz et al.(2007)]{2007AJ....133..439C} Cruz, K.~L., Reid,
  I.~N., Kirkpatrick, J.~D., et al.\ 2007, \aj, 133, 439
   
\bibitem[Fressin et al.(2013)]{fressin2013} Fressin, F., Torres, 
G., Charbonneau, D., et al.\ 2013, \apj, 766, 81 

\bibitem[Ofek(2008)]{2008PASP..120.1128O} Ofek, E.~O.\ 2008, \pasp, 120, 
1128 

\bibitem[Reid et al.(2002)]{2002AJ....124.2721R} Reid, I.~N., Gizis,
  J.~E., \& Hawley, S.~L.\ 2002, \aj, 124, 2721

\bibitem[Reid \& Hawley(2005)]{2005nlds.book.....R} Reid, I.~N., \&
  Hawley, S.~L.\ 2005, New Light on Dark Stars: Red Dwarfs, Low-Mass
  Stars, Brown Dwarfs (Springer-Praxis)

\bibitem[Zheng et al.(2001)]{2001ApJ...555..393Z} Zheng, Z., Flynn,
  C., Gould, A., Bahcall, J.~N., \& Salim, S.\ 2001, \apj, 555, 393

\bibitem[Zheng et al.(2004)]{2004ApJ...601..500Z} Zheng, Z., Flynn,
  C., Gould, A., Bahcall, J.~N., \& Salim, S.\ 2004, \apj, 601, 500

\end{thebibliography}

\appendix{TESS Performance on Kepler Objects of Interest}

Histogram of transit timing precision.

\end{document}
