\documentclass[preprint]{aastex}

\bibliographystyle{apj}

\def \teff {T_{\rm eff}}

\def \phir {\Phi_{\rm R}}

\begin{document}

\title{Simulating the Planet Yield of the Transiting Exoplanet Survey Satellite}

\author{Peter W. Sullivan, Courtney D. Dressing, Josua N. Winn, Roland K. Vanderspek, George R. Ricker}
%\altaffiltext{1}{MIT-Kavli Institute for Astrophysics and Space Research}

%\begin{abstract}

%\end{abstract}

\section{Introduction}
\subsection{Stellar population}
What is the radius function of the local solar neighborhood? (plot of $\phir$)
\subsection{Planet population}
Kepler results and the number of expected (transiting) planets in the nearest $\sim$100 pc. Should we include Petigura rates or Dressing-Charbonneau rates?

\section{Description of TESS}
\subsection{Mission and survey}
Image showing number of pointings vs. ecliptic latitude
\subsection{Optics and CCD}
FOV, EPD, PSF, bandpass, noise. 
Can we say ``as we will present at PDR"?

\section{Simulations}
\subsection{Star Catalog}
How we select from $\phir$, Counts vs. spectral type
\subsection{Background Model}
Zodiacal light, Redo unresolved star analysis?
\subsection{Observing model}
Noise model, observed number of transits, detection thresholds

\section{Results}
\subsection{Planet Yields}
\subsection{Habitability of the TESS-detected planets}
Plot of $S$ vs. $\teff$ for detected planets
\subsection{Observations of the Kepler Objects of Interest}
Histogram of transit timing precision.

\section{Follow-up of the TESS detections}
\subsection{Photometry}
Plot of transits in ppm/hr versus ($I_C$ and/or $J$) magnitude.
\subsection{Radial Velocity}
Plot of RV $K$ vs. ($I_C$ and/or $V$) magnitude.

%\begin{figure}
%\epsscale{1.0}
%\plotone{filters.eps} 
%\caption{\emph{Left:} Intermediate- and narrow-band filters used to detect LAEs. \emph{Right:} Filters used to detect the Ly-$\alpha$ absorbers. For reference, the composite LBG spectrum from \citet{shap03} is also plotted near the expected redshifts of the LAEs and background galaxies.}
%\label{filter}
%\end{figure}

%\section{Conclusions}

\acknowledgements

\clearpage

\begin{thebibliography}{}

\bibitem[Binney \& Merrifield(1998)]{1998gaas.book.....B} Binney, J.,
  \& Merrifield, M.\ 1998, Galactic astronomy (Princeton University
  Press)

\bibitem[Bochanski et al.(2010)]{2010AJ....139.2679B} Bochanski,
  J.~J., Hawley, S.~L., Covey, K.~R., et al.\ 2010, \aj, 139, 2679

%\bibitem[Boyajian et al.(2012)]{2012ApJ...757..112B} Boyajian, T.~S.,
%  von Braun, K., van Belle, G., et al.\ 2012, \apj, 757, 112

\bibitem[Covey et al.(2008)]{2008AJ....136.1778C} Covey, K.~R.,
  Hawley, S.~L., Bochanski, J.~J., et al.\ 2008, \aj, 136, 1778

\bibitem[Cruz et al.(2007)]{2007AJ....133..439C} Cruz, K.~L., Reid,
  I.~N., Kirkpatrick, J.~D., et al.\ 2007, \aj, 133, 439

\bibitem[Reid et al.(2002)]{2002AJ....124.2721R} Reid, I.~N., Gizis,
  J.~E., \& Hawley, S.~L.\ 2002, \aj, 124, 2721

\bibitem[Reid \& Hawley(2005)]{2005nlds.book.....R} Reid, I.~N., \&
  Hawley, S.~L.\ 2005, New Light on Dark Stars: Red Dwarfs, Low-Mass
  Stars, Brown Dwarfs (Springer-Praxis)

\bibitem[Zheng et al.(2001)]{2001ApJ...555..393Z} Zheng, Z., Flynn,
  C., Gould, A., Bahcall, J.~N., \& Salim, S.\ 2001, \apj, 555, 393

\bibitem[Zheng et al.(2004)]{2004ApJ...601..500Z} Zheng, Z., Flynn,
  C., Gould, A., Bahcall, J.~N., \& Salim, S.\ 2004, \apj, 601, 500

\end{thebibliography}

\end{document}
